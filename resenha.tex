\documentclass[12pt, a4paper]{article}
\usepackage[utf8]{inputenc}
\usepackage[T1]{fontenc}
\usepackage[brazil]{babel}
\usepackage{geometry}
\usepackage{hyperref}
\usepackage{float}
\usepackage{tikz}
\usetikzlibrary{shapes.geometric, arrows}

\geometry{
    a4paper,
    left=2.5cm,
    right=2.5cm,
    top=2.5cm,
    bottom=2.5cm
}

\title{Resenha - D.Pedro I: A história não contada \\ Paulo Rezzuti}
\author{
    Pedro Calligaris Delbem - $N_{USP}$: 5255417 \\
}
\date{Prof. Francisco Rodrigues \\ \today}

\begin{document}

    \maketitle
    \thispagestyle{empty}
    \newpage

    \section*{O mais brasileiro e liberal que um monarca português poderia ser}
    Uma excelente biografia que retrata os fatos da maneira mais imparcial possível. A única crítica cabível é, uma vez que a ``propaganda'' é que a mesma foram utilizadas em cartas inéditas, que fossem destacados mais vezes quais fatos são resultantes das mesmas --- apenas me lembro de uma passagem em que isto é citado e não se refere a um acontecimento relevante.
    
    Ademais, a completude do autor de começar com a origem da dinastia de Bragança e terminar com breves histórias dos descendentes de d.Pedro I torna o livro completamente cativante. Sua narrativa comedida não nos deixa enganar quanto aos erros do proclamador da independência, mas não esquece de salientar seus heroicos feitos.
    
    A leitura também me fez refletir quanto ao ensino no país, uma vez que passei pelos ensinos básico e fundamenta completamente ileso a diversos fatos extremamente relevantes da história do país e do mundo. Bem como, a forma anacrônica em que a história é ensinada nos faz desvalorizar o grande ser humano que foi o primeiro imperador do Brasil.
    
    Faz-se notar como o anacronismo pode nos fazer julgar erroneamente personalidades históricas. Enquanto gerado no absolutismo, d.Pedro I foi o mais liberal quanto poderia ser --- lutou como pode pelo fim da escravidão na Brasil e foi ferrenho defensor da causa constitucional por onde passou. Por outro lado, mesmo que a análise não anacrônica --- em parte --- deva compreender em parte suas traições, por outro, a mesma nos faz julgar como terríveis a humilhação que o mesmo submeteu sua esposa ao assumir os filhos fora do casamento e ao alçar Domitila ao mais alto posto.
    
    Por fim, ainda julgo difícil de entender a confiança legada ao irmão d.Miguel apesar do mesmo já ter tramado um golpe contra o pai. Penso que, talvez, d.Pedro I o tenha escolhido para o casamento com d.Maria II --- e por tanto regente --- para tentar "matar dois coelhos em uma só cajadada" resolvendo o problema de Portugal e se reconciliando com o irmão, ao mesmo tempo. Entretanto, me parece uma postura demasiada inocente.
    
    Concluo que é uma excelente experiência de leitura e recomendo a qualquer um que tenha interesse por história.

\end{document}