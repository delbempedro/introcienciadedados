\documentclass[12pt, a4paper]{article}
\usepackage[utf8]{inputenc}
\usepackage[T1]{fontenc}
\usepackage[brazil]{babel}
\usepackage{indentfirst}
\usepackage{geometry}
\usepackage{hyperref}
\usepackage{float}
\usepackage{tikz}
\usetikzlibrary{shapes.geometric, arrows}

\geometry{
    a4paper,
    left=2.5cm,
    right=2.5cm,
    top=2.5cm,
    bottom=2.5cm
}

\title{Resenha --- D.Pedro I: A história não contada \\ Paulo Rezzuti}
\author{
    Pedro Calligaris Delbem --- $N_{USP}$: 5255417 \\
}
\date{Prof. Francisco Rodrigues \\ \today}

\begin{document}

    \maketitle
    \thispagestyle{empty}
    \newpage

    \section*{\centering{\huge{O mais brasileiro e liberal que um monarca português poderia ser}}}

        \subsection*{Introdução}
            Uma excelente biografia que retrata os fatos da maneira mais imparcial possível. Sempre buscando ressaltar os feitos positivos e negativos da personagem, o autor nos apresenta um d.~Pedro I humano, com suas virtudes e defeitos, mas que jamais deixa de ser o protagonista da independência do Brasil, de sua primeira constituição e do enfrentamento ao absolutismo em Portugal.

        \subsection*{Estrutura da Obra}

            A estratégia do autor de introduzir a história da dinastia de Bragança e o contexto político europeu antes de iniciar a narrativa da vida de d.~Pedro I é extremamente válida, uma vez que nos permite compreender melhor as motivações do personagem e o cenário em que os fatos ocorreram. Nesta parte do livro se faz possível compreender o quão ``frágil'' a história é, já que pequenas decisões tomadas por personagens históricos podem ter levado a rumos completamente diferentes. Somam-se a isto diversas curiosidades, como o fato de um ancestral de d.~João VI ter sido ``amaldiçoado'' por um padre que teria dito que nunca os primogênitos da linhagem de Bragança reinariam. Curiosamente, a maldição se confirmou.
    
            Por outro lado, o livro é vendido com a ênfase de que traz cartas inéditas relacionadas ao imperador, mas não deixa claro quais fatos são provenientes destas. Apenas me lembro de uma passagem em que o autor cita uma carta que advém desta nova fonte, mas não se refere a um acontecimento relevante. Fico com a impressão de que o ``marketing'' em cima deste ponto foi um tanto exagerado ou então que o autor não soube demonstrar devidamente a influência destas cartas na narrativa.

            Além disso, a completude do autor de finalizar o livro com breves histórias dos descendentes de d.Pedro I torna o livro completamente cativante. A história que mais me chamou a atenção foi a de um filho bastardo que teve todas as suas posses roubadas quando chegava aos Estados Unidos e que, mesmo assim, conseguiu se reerguer e tornar-se um empresário de sucesso. ELe só veio a descobrir sua verdadeira identidade anos depois, com a morte de seu pai.
    
        \subsection*{Sobre a Personagem}

            A imparcialidade do autor é notória, porquanto não se furta de deixar claro que d.~Pedro I, apesar de se dizer constantemente um liberal, tomou muitas atitudes absolutistas provavelmente devido à sua criação no seio da monarquia portuguesa. Por outro lado, o autor não deixa de ressaltar os feitos do imperador, como a luta pela independência do Brasil e a defesa da constituição tanto no Brasil quanto em Portugal.

            Um exemplo claro desta contradição da figura é o combate à escravidão. Apesar de ter sido um ferrenho defensor do fim da escravidão no Brasil, d.~Pedro I possuía escravos em sua fazenda no Rio de Janeiro. O autor não deixa de ressaltar este fato, mas o contextualiza dentro do cenário da época, em que a abolição total da escravidão era algo praticamente impossível de ser alcançado. Assim, d.~Pedro I é retratado como um homem que lutou pelo que era possível dentro do cenário político e social da época.

            Outro exemplo notório é o fato de a Constituição de 1824 ter sido outorgada por d.~Pedro I, ou seja, sem a participação da população. O autor ressalta que, apesar de ser um ato contrário aos princípios liberais, a constituição foi um avanço significativo para o Brasil, visto que estabeleceu direitos e garantias fundamentais para a população que a versão anterior proposta pela Assembleia Constituinte não contemplava.

            Outro ponto que me fez pensar foi entender a confiança depositada no irmão d.~Miguel apesar de ele já ter tramado um golpe contra o pai. Penso que, talvez, d.~Pedro I o tenha escolhido para o casamento com d.~Maria II --- e, portanto, regente de Portugal --- para tentar ``matar dois coelhos em uma só cajadada'' resolvendo o problema de Portugal e se reconciliando com o irmão, ao mesmo tempo. Entretanto, parece-me uma postura demasiado inocente.

            Já quanto à brasilidade da personagem, fica claro que d.~Pedro I se via como brasileiro e que sua ligação com o país era muito forte. Ele chegou, até mesmo, a considerar que a Assembleia Portuguesa recusaria-o como regente de Portugal, por considerá-lo brasileiro demais. É curioso ressaltar que, talvez, este sentimento de brasilidade tenha se originado da morte de seu filho João Carlos, que foi resultado de uma viagem forçada que Leopoldina e os filhos tiveram que fazer para fugir de um conflito gerado por portugueses exaltados que clamavam pela volta de d.~Pedro a Portugal. Sua ligação com o Brasil foi, assim, reforçada pela perda do filho.

        \subsection*{Reflexões Pessoais}

            A leitura também me fez refletir sobre o ensino no nosso país, pois passei pelos ensinos infantil, fundamental e médio completamente ileso a diversos fatos extremamente relevantes da história do país e do mundo, além da forma anacrônica que a história é ensinada. Um exemplo claro disto é que, à época, homens terem amantes era algo não só socialmente aceito, mas talvez mesmo até um alívio para a mulher, que não precisava se preocupar com novas gravidezes, tendo em vista que não havia métodos contraceptivos eficazes na época.

            Uma característica que me chamou a atenção foi a valorização da educação por parte de d.~Pedro I. Justamente por não ter se educado adequadamente, ele via os estudos como algo fundamental para o desenvolvimento pessoal e social. Desta forma, ele sempre incentivou a educação de seus filhos --- legítimos ou não --- e de seus súditos, buscando sempre promover o acesso ao conhecimento, além de ter colocado na Constituição de 1824 a obrigatoriedade da educação primária para todos os cidadãos brasileiros.
    
        \subsection*{Conclusão}
    
            Faz-se notar como o anacronismo pode nos fazer julgar erroneamente personalidades históricas. Enquanto gerado no absolutismo, d.~Pedro I foi tão liberal quanto poderia ser: lutou como pôde pelo fim da escravidão no Brasil e foi ferrenho defensor da causa constitucional por onde passou.
            
            Embora a análise não anacrônica --- em parte --- deva compreender suas traições, também nos faz julgar como terríveis a humilhação à qual o imperador submeteu sua esposa ao assumir os filhos fora do casamento e ao alçar Domitila ao mais alto posto.
    
            Concluo que é uma excelente experiência de leitura e recomendo a qualquer um que tenha interesse por história e especificamente para aqueles que desejam entender o porquê de o Brasil ser como é.

\end{document}